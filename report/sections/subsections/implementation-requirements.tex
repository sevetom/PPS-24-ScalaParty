\subsection{Requisiti di implementazione} \label{subsec:implementation_requirements}
I requisiti di implementazione del sistema sono requisiti che riguardano le tecnologie, gli strumenti e le pratiche di sviluppo
utilizzati per realizzare il sistema. Questi requisiti sono fondamentali per garantire che il sistema sia sviluppato in modo
efficiente, mantenibile e scalabile in modo tale da soddisfare i requisiti funzionali e non funzionali definiti in precedenza.
I requisiti di implementazione del sistema sono:
\begin{itemize}
    \item \textbf{Tecnologie}, il sistema viene realizzato utilizzando come linguaggio principale Scala e come 
    linguaggio "secondario" Prolog, per la gestione della logica di gioco e delle regole del gioco stesso; inoltre si utilizzano
    SBT come strumento di build e gestione delle dipendenze, IntelliJ IDEA come ambiente di sviluppo integrato (IDE) e Git come
    sistema di controllo versione (in particolare GitHub come repository remoto)
    \item \textbf{Testing}, il sistema deve essere testato in modo tale da garantire che tutte le funzionalità siano implementate
    correttamente e perciò si utilizzano test automatizzati come ScalaTest e JUnit.
\end{itemize}