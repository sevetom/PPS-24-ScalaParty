\subsection{Requisiti Funzionali} \label{sec:functional-requirements}
Questa sezione descrive i requisiti funzionali dell'applicazione Scala Party.
Ogni requisito è stato pensato con l'obiettivo di specificare il comportamento atteso
del gioco durante una partita, sono suddivisi tra ciò che è visibile all'utente e
le funzionalità interne del sistema. Ogni parte è un elenco di requisiti considerati
necessari affinché l'applicazione funzioni come previsto.

\subsubsection{Utente} \label{sec:functional-requirements:user}
Funzionalità accessibili dall'utente:
\begin{itemize}
      \item L'utente può iniziare una nuova partita.
      \item L'utente può lanciare un dado, all'inizio della partita, per determinare il 
            primo giocatore del turno.
      \item L'utente può in ogni momento visualizzare la posizione attuale di ogni giocatore
            sulla plancia.
      \item L'utente può lanciare un o due dadi, durante il proprio turno, per muovere il proprio
            personaggio.
      \item L'utente, durante il movimento, può scegliere la direzione da prendere.
      \item L'utente può in ogni momento visualizzare la posizione delle tessere speciali sulla
            plancia che possono assegnare un piolo o delle monadi.
      \item L'utente può in ogni momento visualizzare il costo per ottenere il piolo.
      \item L'utente può ottenere automaticamente una monade quando finisce sulla tessera 
            corrispondente.
      \item L'utente può ottenere automaticamente un piolo quando finisce sulla tessera 
            corrispondente se possiede sufficienti monadi da scambiare.
      \item L'utente può visualizzare in ogni momento il numero di pioli necessari per vincere 
            il gioco.
      \item L'utente può in ogni momento visualizzare il numero di pioli e monadi raccolti 
            da ciascun giocatore.
      \item L'utente può partecipare a un minigioco casuale alla fine di ogni ciclo di turni.
      \item L'utente può visualizzare le regole di ogni minigioco prima dell'inizio di esso.
      \item L'utente può visualizzare il proprio progresso, dipendente dalle regole, durante u
            un minigioco.
      \item L'utente può visualizzare il vincitore di un minigioco al termine di esso.
      \item L'utente può visualizzare il vincitore del gioco al termine di una partita.
\end{itemize}

\subsubsection{Sistema} \label{sec:functional-requirements:system}
Reazioni automatiche del sistema, che non richiedono l'intervento
diretto dell'utente ma sono una risposta a esso o riguardano la logica interna del gioco.
\begin{itemize}
      \item Il sistema può creare una nuova partita.
      \item Il sistema può generare un numero casuale di pioli necessari per vincere il gioco.
      \item Il sistema può simulare, all'inizio della partita, il lancio di un dado a 6 
            facce per determinare il primo giocatore del turno.
      \item Il sistema può gestire l'assegnazione dei turni ai giocatori in modo che possano
            effettuare la loro mossa.
      \item Il sistema può simulare il lancio di dadi a 6 facce per determinare il numero di
            passi che un giocatore potrà compiere.
      \item Il sistema può aggiornare la posizione di ciascun giocatore sulla plancia.
      \item Il sistema può simulare un comportamento competitivo del giocatore avversario 
            all'utente, muovendosi sulla plancia in modo da raccogliere sufficienti monadi
            per vincere i pioli e il gioco.
      \item Il sistema può generare un numero predefinito di tessere monadi e una sola per 
            volta di piolo da posizionare sulla plancia.
      \item Il sistema può mostrare una plancia che soddisfa le regole. 
      \item Il sistema può calcolare il numero di monadi da pagare per ogni piolo.
      \item Il sistema può premiare un giocatore con una monade quando passa sulla tessera 
            corrispondente.
      \item Il sistema può premiare un giocatore, se possiede abbastanza monadi, con un
            piolo quando passa sulla tessera corrispondente, rimuovendo al giocatore il
            numero di monadi necessario per ottenerlo.
      \item Il sistema può aggiornare il numero di pioli e monadi raccolti da ciascun giocatore.
      \item Il sistema può avviare un minigioco dopo che tutti i giocatori hanno concluso
            il proprio turno.
      \item Il sistema può gestire la logica di ogni semplice minigioco, per esempio 
            basati su una griglia a scacchiera con regole predefinite di movimento e punteggio.
      \item Il sistema può mostrare le regole di ogni minigioco prima dell'inizio di esso.
      \item Il sistema può aggiornare i progressi, dipendenti dalle regole, del giocatore 
            durante un minigioco.
      \item Il sistema può determinare il vincitore di un minigioco.
      \item Il sistema può assegnare un dado aggiuntivo valido solo per il turno successivo al
            vincitore di un minigioco, che verrà selezionato come primo giocatore.
      \item Il sistema può determinare il vincitore del gioco in base al numero di pioli raccolti.
\end{itemize}