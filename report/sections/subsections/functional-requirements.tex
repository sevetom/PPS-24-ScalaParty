\subsection{Requisiti Funzionali} \label{sec:functional-requirements}
Questa sezione descrive i requisiti funzionali dell'applicazione Scala Party.
Ogni requisito è stato pensato con l'obiettivo di specificare il comportamento atteso
del gioco durante una partita, sono suddivisi tra ciò che è visibile all'utente e
le funzionalità interne del sistema. Ogni parte è un elenco di requisiti considerati
necessari affinché l'applicazione funzioni come previsto.

\subsubsection{Utente} \label{sec:functional-requirements:user}
Funzionalità accessibili dall'utente:
\begin{itemize}
    \item L'utente può iniziare una nuova partita.
    \item L'utente può vedere la posizione attuale di ogni giocatore sulla plancia.
    \item L'utente può lanciare un dado per muovere il proprio personaggio.
    \item L'utente, durante il movimento, può scegliere la direzione da prendere.
    \item L'utente può visualizzare la posizione delle tessere speciali sulla plancia che
          assegnano un piolo o delle monadi.
    \item L'utente può visualizzare il numero di pioli e monadi raccolti da ciascun giocatore.
    \item L'utente può partecipare a un minigioco.
    \item L'utente può visualizzare le regole di ogni minigioco.
    \item L'utente può visualizzare i progressi di ogni giocatore durante un minigioco.
    \item L'utente può visualizzare il vincitore di un minigioco.
\end{itemize}

\subsubsection{Sistema} \label{sec:functional-requirements:system}
Reazioni automatiche del sistema, che non richiedono l'intervento
diretto dell'utente ma sono una risposta a esso o riguardano la logica interna del gioco.
\begin{itemize}
    \item Il sistema può creare una nuova partita.
    \item Il sistema può gestire l'assegnazione dei turni ai giocatori in modo che possano
          effettuare la loro mossa.
    \item Il sistema può simulare il lancio di dadi a 6 facce per determinare il numero di
          passi che un giocatore può compiere.
    \item Il sistema può aggiornare la posizione di ciascun giocatore sulla plancia.
    \item Il sistema può simulare il comportamento del giocatore avversario all'utente.
    \item Il sistema può generare tessere speciali da posizionare su una plancia predefinita.
    \item Il sistema può premiare un giocatore con delle monadi quando finisce sulla
          tessera corrispondente.
    \item Il sistema può premiare un giocatore, se possiede abbastanza monadi, con un
          piolo quando finisce sulla tessera corrispondente.
    \item Il sistema può aggiornare il numero di pioli e monadi raccolti da ciascun giocatore.
    \item Il sistema può avviare un minigioco dopo che tutti i giocatori hanno concluso
          il proprio turno.
    \item Il sistema può gestire la logica di ogni minigioco.
    \item Il sistema può aggiornare i progressi di ciascun giocatore durante un minigioco.
    \item Il sistema può determinare il vincitore di un minigioco.
    \item Il sistema può assegnare un dado aggiuntivo valido per il successivo turno al
          vincitore di un minigioco.
    \item Il sistema può determinare il vincitore del gioco in base al numero di pioli raccolti.
\end{itemize}