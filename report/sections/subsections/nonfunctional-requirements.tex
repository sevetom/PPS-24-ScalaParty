\subsection{Requisiti non funzionali} \label{subsec:non_functional_requirements}
I requisiti non funzionali del sistema sono:
\begin{itemize}
    \item \textbf{Prestazioni}: il sistema deve essere in grado di garantire buone prestazioni anche con calcoli
    complessi come la creazione della mappa o durante i mingiochi; inoltre il caricamento dell'applicazione dev'essere rapido 
    non superando i 5 secondi
    \item \textbf{Robustezza}: il sistema deve essere in grado di gestire gli errori in modo da non compromettere
    l'esperienza di gioco dell'utente e allo stesso tempo deve saper gestire input errati o non validi senza andare in crash
    o producendo output non deterministici
    \item \textbf{Manutenibilità}: il sistema deve essere facilmente manutenibile, permettendo agli sviluppatori
    di apportare modifiche senza dover riscrivere grandi porzioni di codice grazie anche alla documentazione del codice
    \item \textbf{Portabilità}: il sistema deve essere in grado di funzionare su diverse piattaforme e dispositivi senza richiedere
    modifiche significative al codice sorgente, garantendo un'esperienza utente coerente su tutti i dispositivi
    \item \textbf{Estensibilità}: il sistema deve essere facilmente estendibile, permettendo agli sviluppatori di aggiungere
    nuove funzionalità o modificare quelle esistenti senza dover riscrivere grandi porzioni di codice
    \item \textbf{Facilità d'uso}: il sistema deve essere intuitivo e facile da usare, permettendo agli utenti di comprendere
    rapidamente come interagire con l'applicazione e le sue funzionalità senza richiedere una curva di apprendimento eccessiva
\end{itemize}