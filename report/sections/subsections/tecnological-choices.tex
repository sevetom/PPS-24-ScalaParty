\subsection{Scelte Tecnologiche}
Le scelte tecnologiche in questo progetto sono state guidate
principalmente da requisiti legati alla prova finale del corso e
dall'esperienza del team con i vari strumenti.
\begin{itemize}
    \item \textbf{Linguaggi}\par
    \begin{itemize}
        \item \textbf{Scala}\par
        In quanto è il principale linguaggio di programmazione che abbiamo 
        studiato durante il corso ed è richiesto per la prova finale.
        \item \textbf{Prolog}\par
        In quanto è il linguaggio con cui abbiamo studiato la programmazione
        logica e permette di implementare in modo conciso ed espressivo
        alcune tipologie di algoritmi che ci aspettiamo di utilizzare
        nel progetto.
    \end{itemize}
    \item \textbf{Framework}\par
    \begin{itemize}
        \item \textbf{ScalaSwing}\par
        Scala Swing è stato scelto come framework per la creazione
        dell'interfaccia grafica del progetto in quanto è un framework
        semplice e il team ha già esperienza con Swing in Java. Inoltre 
        non esistono requisiti che richiedono un'interfaccia grafica
        particolarmente complessa.
        \item \textbf{ScalaTest}\par
        ScalaTest è stato scelto come framework di testing in quanto
        permette di scrivere test in modo espressivo.
    \end{itemize}
    \item \textbf{Build Tool}\par
    \begin{itemize}
        \item \textbf{SBT}\par
        SBT è stato scelto in quanto è il build tool di
        riferimento per progetti Scala e offre un'ottima integrazione
        con il linguaggio.
    \end{itemize}\par
    \item \textbf{Versionamento}\par
    \begin{itemize}
        \item \textbf{Git}\par
        Git è stato scelto come sistema di controllo versione in quanto
        il team ha già esperienza con questo strumento e permette 
        di gestire in modo efficace le modifiche al codice sorgente.

        \item \textbf{GitHub}\par
        GitHub è stato scelto come piattaforma di hosting del codice
        sorgente per la sua facilità d'uso e integrazione con Git.
    \end{itemize}
    \item \textbf{Tool di Supporto}\par
    \begin{itemize}
        \item \textbf{IntelliJ IDEA}\par
        IntelliJ IDEA è stato scelto come ambiente di sviluppo integrato
        in quanto è l'IDE utilizzato dal team durante il corso e
        offre un'ottima integrazione con Scala e SBT.
    \end{itemize}
\end{itemize}