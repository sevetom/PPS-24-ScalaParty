\subsection{Requisiti di Business}
Il progetto presentato in questa relazione viene realizzato 
con i seguenti obiettivi principali:
\begin{enumerate}
    \item Consolidamento delle conoscenze acquisite durante il 
    corso\par in particolare:
    \begin{itemize}
        \item \emph{Scala} come linguaggio di Programmazione Funzionale
        \item \emph{Prolog} come linguaggio di Programmazione Logica
        \item Tecniche avanzate di sviluppo software
        \item Test-Driven-Development (TDD)
        \item Tecniche di organizzazione del processo di sviluppo
    \end{itemize}
    \item Conseguimento della prova d'esame del corso, con esito positivo
\end{enumerate}
Il raggiungimento dei suddetti obiettivi è il requisito di business
per il progetto e si riterrà soddisfatto se:
\begin{enumerate}
    \item Il progetto sarà realizzato in modo da dimostrare
    la padronanza dei concetti di Programmazione Funzionale e 
    Logica, utilizzando anche alcuni dei costrutti avanzati
    dei linguaggi Scala e Prolog
    \item L'organizzazione del processo di sviluppo sarà tale da
    garantire la qualità del software prodotto, nei tempi previsti
    \item Il voto finale dell'esame sarà positivo o possibilmente eccellente
\end{enumerate}